\section*{Problem 1 - Attitude Control of Satellite}

The objective of problem 1 is to control attitude of a Satellite. The satellites equations of motions are represented in equation \eqref{eq:dynamics}, equation ?? \todo{find number in fossen} \cite{Fossen2011}. The parameters for this specific Satellite are; $\mathbf{I}_{CG} = mr^2$, $m = 100 kg$, $r = 2.0 m$. 

\begin{subequations}
\label{eq:dynamics}
	\begin{align}
		\dot{\mathbf{q}} = \mathbf{T}_q (\mathbf{q} ) \boldsymbol{\omega} \\
		\mathbf{I}_{CG} \dot{\boldsymbol{\omega}} - \mathbf{S} (\mathbf{I}_{CG} \boldsymbol{\omega} ) \boldsymbol{\omega} & =  \boldsymbol{\tau} \label{eq:omega_dot}
	\end{align}	
\end{subequations}


\todo{Should we write introduction ?}

\subsection*{Problem 1.1}

\subsubsection*{Finding the equilibrium point}

The equilibrium point is defined as the steady-state solution of a system, meaning $\dot{\mathbf{x}} = 0$ with $\mathbf{x}$ being the state of the system.


The equilibrium point $\mathbf{x_0}$ of the closed-loop system $\mathbf{x} = [ \boldsymbol{\epsilon}^T, \boldsymbol{\omega}^T]^T$ corresponding to $\mathbf{q} = [\eta,\epsilon_1, \epsilon_2, \epsilon_3]^T = [1, 0, 0, 0]$ and $\boldsymbol{\tau} = \boldsymbol{0}$ may be found by setting $\boldsymbol{\epsilon} = \mathbf{0}$ and $\dot{\boldsymbol{\omega}} = \mathbf{0}$. The equation for $\boldsymbol{\epsilon}$ and $\boldsymbol{\omega}$ may be found by using equation (2.86) in \cite{Fossen2011} and \eqref{eq:omega_dot}. From equation (2.86) in \cite{Fossen2011} and \eqref{eq:omega_dot} the following equations was found:


\begin{subequations}
\label{eq:x_dot}
	\begin{align}
		\dot{\boldsymbol{\epsilon}} =  [ \eta \mathbf{I}_{3X3} + \mathbf{S}(\boldsymbol{\epsilon}) ] \boldsymbol{\omega} \\
		 \dot{\boldsymbol{\omega}} = \mathbf{I}_{CG}^{-1} [\mathbf{S} (\mathbf{I}_{CG} \boldsymbol{\omega} ) \boldsymbol{\omega} +  \boldsymbol{\tau} ]
	\end{align}	
\end{subequations}

\todo{Write T ad top}

With $\mathbf{S} (\mathbf{I}_{CG}$ being the skewmatrix of $\boldsymbol{\epsilon} = [\epsilon_1, \epsilon_2, \epsilon_3]^\top$. By using equation \eqref{eq:x_dot},   $\mathbf{q} = [1, 0, 0, 0]^\top$ and $\boldsymbol{\tau} = \boldsymbol{0}$ the equilibrium point $\mathbf{x} = [0, 0, 0, 0]^\top$.

\todo{Should we write more about the calculations?}

\subsubsection*{Linearize the equations of motion}

Linearization: 

\begin{equation}
    \begin{aligned}
    \begin{bmatrix}
		\dot{\boldsymbol{\epsilon}} \\
		\dot{\boldsymbol{\omega}}
    \end{bmatrix}
    &=
	\mathbf{f(\mathbf{x},t)} 
	&= 
	\begin{bmatrix}
		\frac{1}{2}[ \sqrt{1-\boldsymbol{\epsilon}^\top \boldsymbol{\epsilon}} \omega_1  - \epsilon_3 \omega_2 + \epsilon_2 \omega_3  ] \\
		\frac{1}{2}[ \epsilon_1 \omega_3 +  \sqrt{1-\boldsymbol{\epsilon}^\top \boldsymbol{\epsilon}} \omega_2  - \epsilon_1 \omega_3 ] \\
		\frac{1}{2}[  - \epsilon_2 \omega_1  + \epsilon_1 \omega_2 + \sqrt{1-\boldsymbol{\epsilon}^\top \boldsymbol{\epsilon}} \omega_3  ] \\
		\tau_1 \frac{1}{mr^2}\\
		\tau_2 \frac{1}{mr^2}\\
		\tau_3 \frac{1}{mr^2}\\
	\end{bmatrix}
	\end{aligned}
\end{equation}


\begin{equation}
	\mathbf{A} = 
	\begin{bmatrix}
		0 & 0 & 0 & \frac{1}{2} & 0   & 0\\
		0 & 0 & 0 & 0   & \frac{1}{2} & 0\\
		0 & 0 & 0 & 0   & 0   & \frac{1}{2}\\
		0 & 0 & 0 & 0 & 0 & 0\\
		0 & 0 & 0 & 0 & 0 & 0\\
		0 & 0 & 0 & 0 & 0 & 0\\
	\end{bmatrix}
\end{equation}

\begin{equation}
    \begin{aligned}
	\mathbf{A}
	&= 
	\begin{bmatrix}
		0 & 0 & 0 & \frac{1}{2} & 0   & 0\\
		0 & 0 & 0 & 0   & \frac{1}{2} & 0\\
		0 & 0 & 0 & 0   & 0   & \frac{1}{2}\\
		0 & 0 & 0 & 0 & 0 & 0\\
		0 & 0 & 0 & 0 & 0 & 0\\
		0 & 0 & 0 & 0 & 0 & 0\\
	\end{bmatrix}
	=
	\begin{bmatrix}
		0 & 0 & 0 & \frac{1}{2} & 0   & 0\\
		0 & 0 & 0 & 0   & \frac{1}{2} & 0\\
		0 & 0 & 0 & 0   & 0   & \frac{1}{2}\\
		0 & 0 & 0 & 0 & 0 & 0\\
		0 & 0 & 0 & 0 & 0 & 0\\
		0 & 0 & 0 & 0 & 0 & 0\\
	\end{bmatrix}
	\end{aligned}
\end{equation}

\begin{equation}
	\mathbf{B} = 
	\begin{bmatrix}
		0 & 0 & 0 & 0 & 0 & 0\\
		0 & 0 & 0 & 0 & 0 & 0\\
		0 & 0 & 0 & 0 & 0 & 0\\
		0 & 0 & 0 & \frac{1}{400} & 0 & 0\\
		0 & 0 & 0 & 0 & \frac{1}{400} & 0\\
		0 & 0 & 0 & 0 & 0 & \frac{1}{400}\\
	\end{bmatrix}
\end{equation}


\subsection*{Problem 1.2}
Answer Problem 1.2 here. Bold words can be written as \textbf{something bold}. It is also possible to create a new section level:
\subsubsection*{Inner Section 1}
\emph{text..}

\subsubsection*{Inner Section 2}
...

\subsection*{Problem 1.3}
Answer Problem 1.3 here. Equation (2) from the assignment can be written as: 
\begin{equation}
  \label{eq:tau}
  \mathbf{\tau} = -\mathbf{K}_d \boldsymbol{\omega} - k_p \boldsymbol{\epsilon}
\end{equation}

\subsection*{Problem 1.4}
The quaternion error can be written as
 \begin{equation}
	 \tilde{\mathbf{q}} := \left[
	 \begin{array}{c}
		 \tilde{\eta} \\
		 \tilde{\epsilon}
	 \end{array}
	 \right] = \bar{\mathbf{q}}_d \otimes \mathbf{q} 
	 \label{eq:q_tilde}
 \end{equation} 

...

% Note that \mathbf can be used for bold letters in math mode (within equations and dollar signs). \boldsymbol can be used to get bold greek letters.  

