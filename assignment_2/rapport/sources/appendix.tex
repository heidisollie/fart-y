\addcontentsline{toc}{section}{Appendix} \label{sec:appendix}
\appendix

\section{MATLAB Code}\label{sec:matlab}

Section contains \texttt{MATLAB}-code this project. Code posted online on BlackBoard are not in the Appendix. 

\subsection{Calculating discrete model, observability matrix and controllability matrix for 4 states} \label{sec:disk_2}
%\lstinputlisting{code/plot_constraint.m}

\begin{lstlisting}[
    language=Matlab,escapechar=|, label={lst:disk_2}, caption={Calculating discrete model for problem 2}]
    %% Model for Ex. 4

    A_c = [ 0 1 0 0 0 0; 
            0 0 -K_2 0 0 0; 
            0 0 0 1 0 0; 
            0 0 -K_1*K_pp -K_1*K_pd 0 0;
            0 0 0 0 0 1; 
            0 0 0 0 -K_3*K_ep -K_3*K_ed ];
    B_c = [ 0 0; 
            0 0; 
            0 0; 
            K_1*K_pp 0;
            0 0; 
            0 K_3*K_ed];
    
    
    %% Forward Euler method
    deltaT = 0.25;
    I = diag([1,1,1,1,1,1]);
    
    A_d = (I + deltaT*A_c);
    B_d = deltaT * B_c;
    
    %% Stabilizible:

    C= ctrb(A_d,B_d)
    rank(C)
    %% Detectable
    
    O = obsv(A_d,C_d)
    rank(O)
    
    %% LQR
    Q = diag([10 2 1 2]);
    R = 1;
    K = dlqr(A_d,B_d,Q,R);   
\end{lstlisting}

\subsection{Calculation of the optimal input sequence with 4 states }

\begin{lstlisting}[
    language=Matlab,escapechar=|, label={lst:main_code_2}, caption={Calculation of the optimal input sequence with 4 states }] 
    
    % TTK4135 - Helicopter lab
    % The code is heavily based on 
    % the hints/template for problem 2.
    % Updated spring 2017, Andreas L. Fl?ten
    
    %% Initialization and model definition
    init; % Init file for our helicopter
    Discrete_model_euler;
    delta_t	= 0.25; % sampling time
    
    % Discrete time system model. x = [lambda r p p_dot]'
    A1 = A_d % From the discrete computation
    B1 = B_d 
    
    % Number of states and inputs
    mx = size(A1,2); % Number of states (number of columns in A)
    mu = size(B1,2); % Number of inputs(number of columns in B)
    
    % Initial values
    x1_0 = pi;                              % Lambda
    x2_0 = 0;                               % r
    x3_0 = 0;                               % p
    x4_0 = 0;                               % p_dot
    x0 = [x1_0 x2_0 x3_0 x4_0]';          % Initial values
    
    % Time horizon and initialization
    N  = 100;                                % Time horizon for states
    M  = N;                                 % Time horizon for inputs
    z  = zeros(N*mx+M*mu,1);                % Initialize z for the whole horizon
    z0 = z;                                 % Initial value for optimization
    
    % Bounds
    ul 	    = -30*pi/180;                   % Lower bound on control - u1
    uu 	    = 30*pi/180;                    % Upper bound on control - u1
    
    xl      = -Inf*ones(mx,1);              % Lower bound on states (no bound)
    xu      = Inf*ones(mx,1);               % Upper bound on states (no bound)
    xl(3)   = ul;                           % Lower bound on state x3
    xu(3)   = uu;                           % Upper bound on state x3
    
    % Generate constraints on measurements and inputs
    [vlb,vub]       = genbegr2(N,M,xl,xu,ul,uu); 
    vlb(N*mx+M*mu)  = 0;                    % We want the last input to be zero
    vub(N*mx+M*mu)  = 0;                    % We want the last input to be zero
    
    % Generate the matrix Q and the vector c (objecitve function weights in the QP problem) 
    Q1 = zeros(mx,mx);
    Q1(1,1) = 1;                             % Weight on state x1
    Q1(2,2) = 0;                            % Weight on state x2
    Q1(3,3) = 0;                             % Weight on state x3 
    Q1(4,4) = 0;                            % Weight on state x4 
    P1 = 1;                                 % Weight on input
    Q = 2*genq2(Q1,P1,N,M,mu);              % Generate Q
    c = zeros(N*mx+M*mu,1);                 % Generate c
    
    %% Generate system matrixes for linear model
    Aeq = gena2(A1,B1,N,mx,mu);           % Generate A
    beq = zeros(M*mx, 1);        	  % Generate b
    beq(1:mx) = A1*x0; % Initial value
    
    %% Solve QP problem with linear model
    tic
    [z,lambda] = quadprog(Q, c, [], [], Aeq, beq, vlb, vub); 
    t1=toc;
    
    % Calculate objective value
    phi1 = 0.0;
    PhiOut = zeros(N*mx+M*mu,1);
    for i=1:N*mx+M*mu
      phi1=phi1+Q(i,i)*z(i)*z(i);
      PhiOut(i) = phi1;
    end
    
    %% Extract control inputs and states
    u  = [z(N*mx+1:N*mx+M*mu);z(N*mx+M*mu)]; |\label{line:extract_vals}| % Control input from solution 
    
    x1 = [x0(1);z(1:mx:N*mx)];              % State x1 from solution
    x2 = [x0(2);z(2:mx:N*mx)];              % State x2 from solution
    x3 = [x0(3);z(3:mx:N*mx)];              % State x3 from solution
    x4 = [x0(4);z(4:mx:N*mx)];              % State x4 from solution
    
    num_variables = 5/delta_t;|\label{line:zero_padding_start}|
    zero_padding = zeros(num_variables,1);
    unit_padding  = ones(num_variables,1);
    
    u   = [zero_padding; u; zero_padding]; 
    x1  = [pi*unit_padding; x1; zero_padding];
    x2  = [zero_padding; x2; zero_padding];
    x3  = [zero_padding; x3; zero_padding];
    x4  = [zero_padding; x4; zero_padding];|\label{line:zero_padding_end}|
    
    %% Plotting
    t = 0:delta_t:delta_t*(length(u)-1); |\label{line:plotting2}|
    input = [t' u]; 
    figure(2)
    hold on;
    subplot(511)
    stairs(t,u),grid
    ylabel('u')
    subplot(512)
    plot(t,x1,'m',t,x1,'mo'),grid
    ylabel('lambda')
    subplot(513)
    plot(t,x2,'m',t,x2','mo'),grid
    ylabel('r')
    subplot(514)
    plot(t,x3,'m',t,x3,'mo'),grid
    ylabel('p')
    subplot(515)
    plot(t,x4,'m',t,x4','mo'),grid
    xlabel('tid (s)'),ylabel('pdot')
\end{lstlisting}

\newpage


\subsection{Calculating controllability, observability matrix and LQR for 6 states }\label{sec:matlab_4_2}

\begin{lstlisting}[
    language=Matlab,escapechar=|, caption={Calculating controllability matrix, observability matrix and LQR for 6 states}, label={lst:matlab_4_2}]

%% Model for Ex. 4
A_c = [0 1 0 0 0 0;
       0 0 -K_2 0 0 0;
       0 0 0 1 0 0;
       0 0 -K_1*K_pp -K_1*K_pd 0 0; 
       0 0 0 0 0 1;
       0 0 0 0 -K_3*K_ep -K_3*K_ed ];
       
B_c = [0 0;
       0 0;
       0 0;
       K_1*K_pp 0;
       0 0;
       0 K_3*K_ed];

%% Forward Euler method
deltaT = 0.25;
I = diag([1,1,1,1,1,1]);

A_d = (I + deltaT*A_c);
B_d = deltaT * B_c;

%% Stabilizable
C = ctrb(A_d,B_d);
rank(C)

%% Detectable
O = obsv(A_d,C_d)
rank(O)

%% Calculate K
Q = diag([20 1 100 2 300000 500]);
R = [1 0; 0 1];
K = dlqr(A_d,B_d,Q,R);   
    
\end{lstlisting}
\newpage

\subsection{Non-linear constraint on elevation }\label{sec:matlab_connstrains}
\begin{lstlisting}[
    language=Matlab,escapechar=|, caption={Non-linear constraint on elevation}]

function [ c, ceq ] = nonlinearconstraints(z)
    
    global N;
    global mx;
    size(N);
    size(mx);
    
    alpha = 0.2;
    beta = 20;
    lambda_t = 2*pi/3;
    c = zeros(N,1);
    for i = 0:(N-1)
        c(i+1) = alpha*exp(-beta*(z(1 +mx*i) - lambda_t)^2) - z(5 +mx*i);
    end

    ceq = [];
    
\end{lstlisting}

\subsection{Calculation of the optimal input sequence with 6 states }\label{sec:matlab_4_3}

\begin{lstlisting} [
    language=Matlab,escapechar=|, label={lst:main_code_4}, caption={Calculation of the optimal input sequence with 6 states}]

% TTK4135 - Helicopter lab
% Hints/template for problem 2.
% Updated spring 2017, Andreas L. Fl?ten

%% Initialization and model definition
init; 
Discrete_model_euler;
delta_t	= 0.25; % sampling time
global N;
global mx;
global M;

% Discrete time system model. 
A1 = A_d; 
B1 = B_d; 

% Number of states and inputs
mx = size(A1,2); % Number of states 
mu = size(B1,2); % Number of inputs

% Initial values
x0 = [ pi 0 0 0 0 0 ]';          

% Time horizon and initialization
N  = 40;            % Time horizon for states
M  = N;             % Time horizon for inputs
z  = zeros(N*mx+M*mu,1);                    
z0 =z; 

% Bounds
pitch_lim = 30*pi/180; |\label{line:start_limit}|
pitch_rate_lim = inf; %20*pi/180;
elevation_lim = inf;
ul =[-pitch_lim; -elevation_lim];
uu =[pitch_lim; elevation_lim];

xl      = [-inf -inf -pitch_lim -pitch_rate_lim -inf -inf]';              % Lower bound on states 
xu      = [inf inf pitch_lim pitch_rate_lim inf inf]';               % Upper bound on states 

% Generate constraints on measurements and inputs
[vlb,vub]       = genbegr2(N,M,xl,xu,ul,uu);|\label{line:end_limit}| ;

% Generate the matrix Q and R and G (objecitve function weights in the QP problem) 
Q1 = diag([1 0 0 0 0 0]);
R = diag([1 1]);
G = 2*genq2(Q1,R,N,M,mu); |\label{line:4_G}|

%% Generate system matrixes for linear model
Aeq = gena2(A1,B1,N,mx,mu) |\label{line:start_eq}|; beq = zeros(M*mx, 1);        
beq(1:mx) = A1*x0 |\label{line:end_eq}|; % Initial value

%% Objective function
f = @(Z) Z'*G*Z;

%% SQP
options = optimoptions('fmincon','Algorithm','sqp','MaxFunEvals',60000);
tic
Z = fmincon(f, z0, [], [], Aeq, beq, vlb, vub, @nonlinearconstraints, options); |\label{line:4_fmincon}|
t1=toc;

%% Extract control inputs and states
u1  = [Z(N*mx+1:mu:N*mx+M*mu); Z(N*mx + M*mu - 1)]; 
u2 = [Z(N*mx+2:mu:N*mx+M*mu); Z(N*mx + M*mu)];


x1 = [x0(1);Z(1:mx:N*mx)];              
x2 = [x0(2);Z(2:mx:N*mx)];              
x3 = [x0(3);Z(3:mx:N*mx)];              
x4 = [x0(4);Z(4:mx:N*mx)];              
x5 = [x0(5);Z(5:mx:N*mx)];              
x6 = [x0(6);Z(6:mx:N*mx)];              

num_variables = 5/delta_t;
zero_padding = zeros(num_variables,1);
unit_padding  = ones(num_variables,1);

u1   = [zero_padding; u1; zero_padding];
u2   = [zero_padding; u2; zero_padding];
u = [u1 u2];
x1  = [pi*unit_padding; x1; zero_padding];
x2  = [zero_padding; x2; zero_padding];
x3  = [zero_padding; x3; zero_padding];
x4  = [zero_padding; x4; zero_padding];
x5  = [zero_padding; x5; zero_padding];
x6  = [zero_padding; x6; zero_padding];
x = [x1 x2 x3 x4 x5 x6];

%% LQR
Q = diag([20 1 100 2 300000 500]);
R = [1 0; 0 1];
K = dlqr(A1,B1,Q,R);      
% 
%% Plotting
t = 0:delta_t:delta_t*(length(u)-1);
input_opt = [t' u]; 
x_opt = [t' x];

\end{lstlisting}



\section{Simulink Diagrams}

\subsection{Diagram from implementation of optimal control of pitch/travel without feedback}

\begin{figure}[!h]
    \centering
	\includegraphics[width=1.2\textwidth]{figures/part2/simulink_lab2.PNG}
	\caption{Simulink diagram for Optimal Control of Pitch/Travel without Feedback}
\label{fig:L2_sim}
\end{figure}

\newpage

\subsection{Diagrams from implementation of optimal control of pitch/travel with feedback}\label{sec:simmulink_3_2}

\begin{figure}[!h]
	\centering
	\includegraphics[width=1.00\textwidth]{figures/part3/im_3_2_full.PNG}
	\caption{Simulink diagram of the system implemented with feedback and LQR-controller.}
    \label{fig:3_2_full}
\end{figure} 

\begin{figure}[!h]
	\centering
	\includegraphics[width=1.00\textwidth]{figures/part3/im_3_2_u.PNG}
	\caption{Feedback implemented in system}
    \label{fig:3_2_u}
\end{figure} 


\newpage

\subsection{Diagram from implementation of optimal control of pitch/travel and elevation with
and without feedback}

\begin{figure}[!h]
    \centering
	\includegraphics[width=1.2\textwidth]{figures/part4/sim_4_open_loop.PNG}
	\caption{Simulink diagram for Optimal Control of Pitch/Travel and Elevation without Feedback}
\label{fig:L4_sim_open}
\end{figure}

\begin{figure}[!h]
    \centering
	\includegraphics[width=1.2\textwidth]{figures/part4/sim_4_closed_loop.PNG}
	\caption{Simulink diagram for Optimal Control of Pitch/Travel and Elevation with Feedback}
\label{fig:L4_sim_closed}
\end{figure}

\begin{figure}[!h]
    \centering
	\includegraphics[width=1.2\textwidth]{figures/part4/sim_4_closed_loop_inside_control.PNG}
	\caption{Inner simulink diagram of the 'Optimal Control' box from \Cref{fig:L4_sim_closed}}
\label{fig:L4_sim_closed_control}
\end{figure}

\newpage